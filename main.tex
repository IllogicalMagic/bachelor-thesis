\documentclass[12pt,a4paper]{article}

\usepackage[top=2.0cm, bottom=2.0cm, left=3cm, right=1.5cm, footskip=0.5cm]{geometry}

\usepackage{cmap}                               % Улучшенный поиск русских слов в полученном pdf-файле
\usepackage[T1,T2A]{fontenc}                    % Поддержка русских букв
\usepackage[utf8]{inputenc}[2014/04/30]         % Кодировка utf8
\usepackage[english, russian]{babel}[2014/03/24]% Языки: русский, английский
\usepackage{setspace}

\begin{document}

% Если содержание расползётся на две страницы
% \doublespacing
\tableofcontents
% \singlespacing
\newpage

\section{Вступление}

Немного о современной архитектуре, embedded, об отходе от ручных оптимизаций в embedded, о продвинутых компиляторных анализах и оптимизациях.

\section{Цель работы}

Вся работа, проведённая в рамках данного проекта, может быть поделена на пять основных частей, которые, в свою очередь, также охватывают несколько различных областей в соответствующих темах.

\paragraph{Исследование алгоритмов}

\subparagraph{Косвенные доступы}

\subparagraph{Рекурсивные структуры данных}

\paragraph{Исследование компиляторов}

\subparagraph{Возможности}

\subparagraph{Опции}

\subparagraph{Прагмы}

\paragraph{Разработка фаз}

\subparagraph{Улучшение фазы для регулярных доступов}

\paragraph{Добавление опций}

\paragraph{Тестирование фаз}

\section{Существующие решения}

Прежде чем приступить к непосредственно разработке, необходимо провести анализ существующих алгоритмов в литературе и публикациях и решений в компиляторах. Стоит отметить, что сама реализация каких-либо алгоритмов в компиляторах обычно представляет собой чёрный ящик ввиду отсутсвия доступа к исходному коду. Поэтому рассматривается только пользовательский интерфейс, предоставленный для управления параметрами оптимизаций.

\subsection{Алгоритмы}

Немного про рдс, определения, различные подходы. Агрессивные оптимизации и не очень.

\subsection{Компиляторы}

Здесь обзор четырёх компиляторов в плане опций и прагм.

\paragraph{GCC}

Существующая фаза и опции. Ссылка на доку.

\paragraph{LLVM}

Существующая фаза и опции. В скобках: см. в таком-то модуле такого-то компилятора.

\paragraph{ICC}

Прагмы. Ссылка на документацию.

\paragraph{Oracle C/C++ Compiler}

Опции. Ссылка на документацию.

\section{Инфраструктура}

В данной работе для реализации использовалась инфраструктура проекта с открытым кодом LLVM, а также одного из его подпроектов -- Clang.

\subsection{LLVM}

Про сам проект, про фронтенд, про лит и LNT.

\subsection{Архитектура ARC}

Небольшой обзор архитектуры: embedded, конфигурируемые процессоры, широко используются в различных областях, интересный набор инструкций.

\section{Реализация}

В ходе работы были реализованы четыре фазы, каждая из которых специализируется на каком-то конкретном виде доступа к данным

\subsection{Преаллокация}

Вводные слова о том, что тут вообще происходит и кто всё придумал.

\subsubsection{Архитектурные особенности}

Немного про саму инструкцию.

\subsubsection{Алгоритм}

Подробное описание алгоритма, который я хотел имплементить, со всеми опасностями и ограничениями.

\subsubsection{Возможности для улучшения}

А здесь уже ослабление требований, введение каких-то новых прагм, атрибутов, использование анализа алиасов и т.д.

\subsection{Регулярные доступы}

Примерчики, рассказ о том, что в некоторых архитектурах оптимизация не имеет смысла из-за наличия префетчера в железе.

\subsubsection{Существующая фаза}

Небольшое описание существующей фазы и её возможностей.

\subsubsection{Добавленные возможности}

То, что конкретно я улучшил в этой фазе. Распространение на внешние циклы и добавление предподкачки перед циклом.

\subsubsection{Возможности для улучшения}

Дополнительный анализ на применимость, учёт профильной информации.

\subsection{Косвенные доступы}

Что есть косвенный доступ, почему важно.

\subsubsection{Базовый алгоритм}

Здесь рассказ об алгоритме товарищей из Кэмбриджа.

\subsubsection{Расширенная версия}

А здесь уже моя модификация и расширение этого алгоритма.

\paragraph{Поиск базовых индуктивных переменных}

\paragraph{Случаи с шагом большим единицы и неизвестным на этапе компиляции шагом}

\paragraph{Вычисление дистанции с учётом сгенерированного кода}

\subsubsection{Дальнейшие улучшения}

Анализ указателей, профильная информация, более аккуратная генерация инструкций, возможность использовать скалярную эволюцию для промежуточных вычислений, учёт давления на регистры.

\subsection{Рекурсивные структуры данных}

Рассказ о работах Тодда Маури и Ченга Люка. Определение различных типов указателей в структурах. Различные методики для предподкачки.

\subsubsection{Определение рекурсивных структур данных}

Возможно, слить с тем, что сверху, либо переделать структуру.

\subsubsection{Алгоритм}

Конкретно моя реализация.

\subsubsection{Дальнейшие улучшения}

Профильная информация, менее жадный префетч, перестановка инструкций местами.

\section{Контроль фаз}

Почему важно давать пользователю ручки.

\subsection{Опции}

Здесь про изучение опций в компиляторах.

\paragraph{Sun Compiler}

\paragraph{GCC}

\paragraph{LLVM}

\paragraph{Добавленные опции}

Мои наработки.

\subsection{Прагмы}

Здесь о прагмах.

\paragraph{ICC}

\paragraph{Добавленные прагмы}

\subsection{Аттрибуты}

Немного о том, зачем это вообще пригодилось и почему нет в других компиляторах.

\paragraph{Добавленный атрибут}

\section{Тестирование}

Что именно было важно в данной работе: отладка и применимость.

\subsection{Инфраструктура для тестирования}

Немного про тулзы, которые использовались, типа лита, метаваре тулчейна, симулятора, нативной платформы.

\paragraph{Ливерморские циклы}

Что это, чем знаменито.

\paragraph{LLVM Nightly Testsuite}

Хорошее описание, из чего состоит и зачем используется.

\paragraph{Тесты для симулятора}

Упоминание про DSP-stone и ручные тесты.

\paragraph{Юнит-тесты}

Мелкие тесты для лита и для x86 запуска.

\subsection{Применимость фаз}

Здесь красиво о количестве применений разных фаз. Предположения о том, почему так.

\subsection{Покрытие кода}

Расписать наборы тестов, почему именно такой выбор. Далее красиво, с процентиками, покрытие. Объяснение, почему хорошее, но не идеальное (по крайней мере, функциональное).

\section{Результат работы}

Здесь копия целей, но уже выполненных в результате работы.

\section{Заключение}

А вот здесь немного общих слов оценивающего характера о моей работе. После этого дальнейшие направления - улучшение статического анализа, учёт профильной информации и предподкачка кода.

\end{document}
