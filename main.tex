\documentclass[12pt,a4paper]{article}

\usepackage{cmap}                               % Улучшенный поиск русских слов в полученном pdf-файле
\usepackage[T1,T2A]{fontenc}                    % Поддержка русских букв
\usepackage[utf8]{inputenc}[2014/04/30]         % Кодировка utf8
\usepackage[english, russian]{babel}[2014/03/24]% Языки: русский, английский
\title{My first document}
\date{2018-22-06}
\author{Me}

\begin{document}
\pagenumbering{gobble}

\tableofcontents
\newpage

\pagenumbering{arabic}

\section{Вступление}

\section{Цель работы}

\paragraph{Исследование алгоритмов}

\subparagraph{Косвенные доступы}

\subparagraph{Рекурсивные структуры данных}

\paragraph{Исследование компиляторов}

\subparagraph{Возможности}

\subparagraph{Опции}

\subparagraph{Прагмы}

\paragraph{Разработка фаз}

\subparagraph{Улучшение фазы для регулярных доступов}

\paragraph{Добавление опций}

\paragraph{Тестирование фаз}

\section{Существующие решения}

\subsection{Алгоритмы}

\subsection{Компиляторы}

\section{Инфраструктура}

\subsection{LLVM}

\subsection{Архитектура ARC}

\section{Реализация}

\subsection{Преаллокация}

\subsubsection{Архитектурные особенности}

\subsubsection{Алгоритм}

\subsubsection{Возможности для улучшения}

\subsection{Регулярные доступы}

\subsubsection{Существующая фаза}

\subsubsection{Добавленные возможности}

\subsubsection{Возможности для улучшения}

\subsection{Косвенные доступы}

\subsubsection{Базовый алгоритм}

\subsubsection{Расширенная версия}

\paragraph{Поиск базовых индуктивных переменных}

\paragraph{Случаи с шагом большим единицы и неизвестным на этапе компиляции шагом}

\paragraph{Вычисление дистанции с учётом сгенерированного кода}

\subsubsection{Дальнейшие улучшения}

\subsection{Рекурсивные структуры данных}

\subsubsection{Определение рекурсивных структур данных}

\subsubsection{Алгоритм}

\section{Контроль фаз}

\subsection{Опции}

\paragraph{Sun Compiler}

\paragraph{GCC}

\paragraph{LLVM}

\paragraph{Добавленные опции}

\subsection{Прагмы}

\paragraph{ICC}

\paragraph{Добавленные прагмы}

\subsection{Аттрибуты}

\paragraph{Добавленный атрибут}

\section{Тестирование}

\subsection{Инфраструктура для тестирования}

\paragraph{Ливерморские циклы}

\paragraph{LLVM Nightly Testsuite}

\paragraph{Тесты для симулятора}

\paragraph{Юнит-тесты}

\subsection{Применимость фаз}

\subsection{Покрытие кода}

\section{Результат работы}

\section{Заключение}

\end{document}
